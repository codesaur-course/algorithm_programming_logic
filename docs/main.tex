\documentclass{report}
\usepackage[utf8]{inputenc}
\usepackage[brazil]{babel}

\title{Programação para Iniciantes \\ \large 1ª Edição}
\author{Ralph dos Santos Souza}
\date{\today}

\begin{document}

\maketitle

\tableofcontents

\chapter*{Introdução}
\addcontentsline{toc}{chapter}{Introdução}
\begin{center}
    Antes de começarmos permitam-me dar-lhes as boas vindas ao Programação para Iniciantes. Este livro tem como objetivo
    apresentar os conhecimentos básicos no mundo da Tecnologia da Informação para àqueles que nunca antes se aventuraram por aqui
    e que agora buscam trilhar este caminho.\\
    \vspace{12pt}
    Este livro surgiu da minha própria experiência, no início da minha jornada, quando comecei a entender alguns pontos e conceitos
    importantes da área, coisas que uma pessoa que nunca teve qualquer contato com a área jamais teria como conhecer. 
    É claro que assim como eu consegui entender os mesmos qualquer pessoa conseguiria mas, por quê procurar informações distintas 
    em lugares distintos se poderemos fazer o mesmo através desta obra?\\
    \vspace{12pt}
    Aqui nós iremos entender o que é uma \textbf{stack}, qual a melhor aplicação práticas para certas linguagens, o que são \textbf{frameworks}, 
    quais são os papéis desempenhados pelo \textbf{backend} e pelo \textbf{frontend}, o que é um programador \textbf{fullstack}, o que faz
    um \textbf{QA (Quality Assurance)}, qual o papel do \textbf{DevOps}, qual o papel do \textbf{DBA (Database Analist)}, entre muitas outras
    coisas.\\
    \vspace{12pt}
    É claro que não iremos nos aprofundar em cada um desses pontos, afinal de contas cada um é um universo à parte, infinito em conhecimento 
    e informação. A proposta aqui é outra, é permitir que você, que busca ingressar nessa área, possa entender um pouco mais, de maneira
    prática e quem sabe divertida, sobre os desafios ocultos do TI.
\end{center}

\chapter*{Agradecimentos}
\addcontentsline{toc}{chapter}{Agradecimentos}
\begin{center}
    
\end{center}

\chapter{}

\chapter{}

\end{document}